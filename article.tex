%%%%%%%%%%%%%%%%%%%%%%%%%%%%%%%%%%%%%%%%%
% Journal Article
% LaTeX Template
% Version 1.4 (15/5/16)
%
% This template has been downloaded from:
% http://www.LaTeXTemplates.com
%
% Original author:
% Frits Wenneker (http://www.howtotex.com) with extensive modifications by
% Vel (vel@LaTeXTemplates.com)
%
% License:
% CC BY-NC-SA 3.0 (http://creativecommons.org/licenses/by-nc-sa/3.0/)
%
%%%%%%%%%%%%%%%%%%%%%%%%%%%%%%%%%%%%%%%%%

%----------------------------------------------------------------------------------------
%	PACKAGES AND OTHER DOCUMENT CONFIGURATIONS
%----------------------------------------------------------------------------------------

\documentclass[10pt, a4paper, twocolumn]{article}
%%%%%%%%%%%%%%%%%%%%%%%%%%%%%%%%%%%%%%%%%
% Wenneker Article
% Structure Specification File
% Version 1.0 (28/2/17)
%
% This file originates from:
% http://www.LaTeXTemplates.com
%
% Authors:
% Frits Wenneker
% Vel (vel@LaTeXTemplates.com)
%
% License:
% CC BY-NC-SA 3.0 (http://creativecommons.org/licenses/by-nc-sa/3.0/)
%
%%%%%%%%%%%%%%%%%%%%%%%%%%%%%%%%%%%%%%%%%

%----------------------------------------------------------------------------------------
%	PACKAGES AND OTHER DOCUMENT CONFIGURATIONS
%----------------------------------------------------------------------------------------

\usepackage[english]{babel} % English language hyphenation

\usepackage{microtype} % Better typography

\usepackage{amsmath,amsfonts,amsthm} % Math packages for equations

\usepackage[svgnames]{xcolor} % Enabling colors by their 'svgnames'

\usepackage[hang, small, labelfont=bf, up, textfont=it]{caption} % Custom captions under/above tables and figures

\usepackage{booktabs} % Horizontal rules in tables

\usepackage{lastpage} % Used to determine the number of pages in the document (for "Page X of Total")

\usepackage{graphicx} % Required for adding images

\usepackage{enumitem} % Required for customising lists
\setlist{noitemsep} % Remove spacing between bullet/numbered list elements

\usepackage{sectsty} % Enables custom section titles
\allsectionsfont{\usefont{OT1}{phv}{b}{n}} % Change the font of all section commands (Helvetica)

%----------------------------------------------------------------------------------------
%	MARGINS AND SPACING
%----------------------------------------------------------------------------------------

\usepackage{geometry} % Required for adjusting page dimensions

\geometry{
	top=1cm, % Top margin
	bottom=1.5cm, % Bottom margin
	left=2cm, % Left margin
	right=2cm, % Right margin
	includehead, % Include space for a header
	includefoot, % Include space for a footer
	%showframe, % Uncomment to show how the type block is set on the page
}

\setlength{\columnsep}{7mm} % Column separation width

%----------------------------------------------------------------------------------------
%	FONTS
%----------------------------------------------------------------------------------------

\usepackage[T1]{fontenc} % Output font encoding for international characters
\usepackage[utf8]{inputenc} % Required for inputting international characters

\usepackage{XCharter} % Use the XCharter font

%----------------------------------------------------------------------------------------
%	HEADERS AND FOOTERS
%----------------------------------------------------------------------------------------

\usepackage{fancyhdr} % Needed to define custom headers/footers
\pagestyle{fancy} % Enables the custom headers/footers

\renewcommand{\headrulewidth}{0.0pt} % No header rule
\renewcommand{\footrulewidth}{0.4pt} % Thin footer rule

\renewcommand{\sectionmark}[1]{\markboth{#1}{}} % Removes the section number from the header when \leftmark is used

%\nouppercase\leftmark % Add this to one of the lines below if you want a section title in the header/footer

% Headers
\lhead{} % Left header
\chead{\textit{\thetitle}} % Center header - currently printing the article title
\rhead{} % Right header

% Footers
\lfoot{} % Left footer
\cfoot{} % Center footer
\rfoot{\footnotesize Page \thepage\ of \pageref{LastPage}} % Right footer, "Page 1 of 2"

\fancypagestyle{firstpage}{ % Page style for the first page with the title
	\fancyhf{}
	\renewcommand{\footrulewidth}{0pt} % Suppress footer rule
}

%----------------------------------------------------------------------------------------
%	TITLE SECTION
%----------------------------------------------------------------------------------------

\newcommand{\authorstyle}[1]{{\large\usefont{OT1}{phv}{b}{n}\color{DarkRed}#1}} % Authors style (Helvetica)

\newcommand{\institution}[1]{{\footnotesize\usefont{OT1}{phv}{m}{sl}\color{Black}#1}} % Institutions style (Helvetica)

\usepackage{titling} % Allows custom title configuration

\newcommand{\HorRule}{\color{DarkGoldenrod}\rule{\linewidth}{1pt}} % Defines the gold horizontal rule around the title

\pretitle{
	\vspace{-30pt} % Move the entire title section up
	\HorRule\vspace{10pt} % Horizontal rule before the title
	\fontsize{32}{36}\usefont{OT1}{phv}{b}{n}\selectfont % Helvetica
	\color{DarkRed} % Text colour for the title and author(s)
}

\posttitle{\par\vskip 15pt} % Whitespace under the title

\preauthor{} % Anything that will appear before \author is printed

\postauthor{ % Anything that will appear after \author is printed
	\vspace{10pt} % Space before the rule
	\par\HorRule % Horizontal rule after the title
	\vspace{20pt} % Space after the title section
}

%----------------------------------------------------------------------------------------
%	ABSTRACT
%----------------------------------------------------------------------------------------

\usepackage{lettrine} % Package to accentuate the first letter of the text (lettrine)
\usepackage{fix-cm}	% Fixes the height of the lettrine

\newcommand{\initial}[1]{ % Defines the command and style for the lettrine
	\lettrine[lines=3,findent=4pt,nindent=0pt]{% Lettrine takes up 3 lines, the text to the right of it is indented 4pt and further indenting of lines 2+ is stopped
		\color{DarkGoldenrod}% Lettrine colour
		{#1}% The letter
	}{}%
}

\usepackage{xstring} % Required for string manipulation

\newcommand{\lettrineabstract}[1]{
	\StrLeft{#1}{1}[\firstletter] % Capture the first letter of the abstract for the lettrine
	\initial{\firstletter}\textbf{\StrGobbleLeft{#1}{1}} % Print the abstract with the first letter as a lettrine and the rest in bold
}

%----------------------------------------------------------------------------------------
%	BIBLIOGRAPHY
%----------------------------------------------------------------------------------------



\usepackage[utf8]{inputenc}

\usepackage{blindtext} % Package to generate dummy text throughout this template 
\usepackage[T1]{fontenc} % Use 8-bit encoding that has 256 glyphs

\usepackage[english]{babel} % Language hyphenation and typographical rules

\usepackage{abstract} % Allows abstract customization

%\usepackage{titlesec} % Allows customization of titles
%\renewcommand\thesection{\Roman{section}} % Roman numerals for the sections
%\renewcommand\thesubsection{\roman{subsection}} % roman numerals for subsections
%\titleformat{\section}[block]{\large\scshape\centering}{\thesection.}{1em}{} % Change the look of the section titles
%\titleformat{\subsection}[block]{\large}{\thesubsection.}{1em}{} % Change the look of the section titles



\usepackage{hyperref} % For hyperlinks in the PDF

%--------------------------------------------------
% LIBRARY SECTION



%--------------------------------------------------
%----------------------------------------------------------------------------------------
%	TITLE SECTION
%----------------------------------------------------------------------------------------



%\pretitle{\begin{center}\Huge\bfseries} % Article title formatting
%\posttitle{\end{center}} % Article title closing formatting
\title{Agile Methods and Model Driven Engineering: a survey}% Article title
\author{%
\authorstyle{Faustin Ahishakiye\textsuperscript{1}, Angela Barriga Rodriguez\textsuperscript{1}, Frikk Hosøy Fossdal\textsuperscript{1}, Job Nyangena\textsuperscript{2}, Patrick Stünkel\textsuperscript{1} and Alejandro Rodriguez Tena\textsuperscript{1}
}\\
\newline\newline % Space before institutions
\textsuperscript{1}\institution{Western Norway University of Applied Sciences}\\
\textsuperscript{2}\institution{University of Bergen}
}
\date{\today} % Leave empty to omit a date


%----------------------------------------------------------------------------------------

\begin{document}

% Print the title
\maketitle

\thispagestyle{firstpage}


\lettrineabstract{Lorem ipsum dolor sit amet, TODO ARTE writes the abstract consectetur adipiscing elit. Fusce maximus nisi ligula. Morbi laoreet ex ligula, vitae lobortis purus mattis vel. Vestibulum ante ipsum primis in faucibus orci luctus et ultrices posuere cubilia Curae; Donec ac metus ut turpis mollis placerat et nec enim. Duis tristique nibh maximus faucibus facilisis. Praesent in consequat leo. Maecenas condimentum ex rhoncus, elementum diam vel, malesuada ante.} %TODO ARTE
%----------------------------------------------------------------------------------------
%	ARTICLE CONTENTS
%----------------------------------------------------------------------------------------

\section{Introduction}

% Really nice looking first letter
%\lettrine[nindent=0em,lines=3]{L} orem ipsum dolor sit amet, consectetur adipiscing elit.

TODO FAHI write the intro %TODO FAHI

\section{Method}

TODO PAST write the method %TODO PAST

\section{Survey}

%TODO FAHI, FFO, JOB wrtie their summary
\subsection{Agile Model Driven Development}

Modeling is an important part of all software development projects because it enables to think through complex issues before the attempt to address them via code. This is true for agile projects, for not-so-agile projects, for embedded projects, and for business application projects. Unfortunately, many modeling efforts prove to be dysfunctional. At one end of the spectrum there are projects where no modeling is performed, either because the developers haven’t any modeling skills or because they have abandoned modeling as a useless endeavor. At the other end of the spectrum there are projects which sink in a morass of documentation and overly detailed models, either because the project team suffers from “analysis paralysis” and finds itself unable to move forward or because the team has burdened itself with too many modeling specialists who don’t have the skills to move forward even if they wanted to. Somewhere in the middle, there are project teams that invest in modeling and documentation efforts only to discover that the programmers ignore the models anyway, often because the models are unrealistic or simply because the programmers think they know better than the modelers (and often they do). The goal of Agile Model Driven Development (AMDD) is to show how to avoid these problems, to gain the benefits of modeling and documentation without suffering the drawbacks \cite{1}.

AMDD as the agile version of MDD (Model Driven Development) takes a much more realistic approach: its goal is to describe how developers and stakeholders can work together cooperatively to create models which are just barely good enough. It assumes that each individual has some modeling skills, or at least some domain knowledge, that they will apply together in a team in order to get the job done. It is reasonable to assume that developers will understand a handful of the modeling techniques out there, but not all of them. It is also reasonable to assume that people are willing to learn new techniques over time, often by working with someone else that already has those skills.

AMDD does not require everyone to be a modeling expert, it just requires them to be willing to try. AMDD also allows people to use the most appropriate modeling tool for the job, often very simple tools such as whiteboards or paper, because one wants to find ways to communicate effectively, not document comprehensively. There is nothing wrong with sophisticated CASE tools in the hands of people who know how to use them, but AMDD does not depend on such tools.

\subsection{Agile Model Driven Development Is Good Enough}

S. W. Ambler \cite{2} believes that modeling is a way to think issues through before the code step because it lets to think at a higher abstraction level. One can also do this by writing a test before writing functional code, along the lines of test-driven development.
With the Agile Model Driven Development (AMDD) approach, one typically do just enough high-level modeling at the beginning of a project to understand the scope and potential architecture of the system, and then during development iterations modelling will be done as part of the iteration planning activities to. In this way light models will be done in several minutes as precursor to several hours of coding.

An agile model is just barely good enough—it meets its goals and no more. Several times experts tend to use complex models, even when the situation can be described in a softer and simpler way. The vast majority of models can be drawn on a whiteboard, on paper or even the back of a napkin. Whenever one of these diagrams want to be saved it can be taken a picture of it with a digital camera, or even simply transcribe it onto paper. This works because most diagrams are throwaways; their true value comes from drawing them to think through an issue, and once the issue is resolved the diagram doesn't offer much value.

In the end of \cite{2} there is a discussion between Axel Uhl and Scott W Ambler, Axel remarks some projects where MDA succeeded nevertheless Scott answers saying that he is not very excited about MDA due to big failures of some projects in the past.

\subsection{Agile Model Driven Development: An Intelligent Compromise}

This article \cite{3} presents the concept of Agile Model Driven Development or AMDD as an attempt to effectively bring together the fast pace of agile development and the guaranteed quality of model-driven development. This methodology makes use of the strong contrast between Agility and Model Driven Software Engineering or MDSE, attempting to cover each of their flaws with the strong points of the other, creating a symbiotic association. For instance, MDSE strict focus on documentation and models generation can supply perfectly the documentation “phobia” that suffers agile development.

The author reviews four different types of AMDD processes used in academia and industry:
\begin{itemize}
\item Sage (2006): uses a MDD-Based approach, with the main objective of applying agility to high assurance software. Its main contribution has been the support of executable delivery from partial conflicting models.
\item Hybrid MDD (2009): Assembly-Based, focus on applying MDD to small and middle size projects. It contributed with the support of the partial usage of MDD activities in collaboration with traditional programming practices.
\item MDD-SLAP (2011): Agile-Based, focused on benefiting from both agile and MDD advantages in developing real-time telecommunication systems. Its main contribution is establishing a simple, yet fundamental correspondence between MDD activities and agile practices.
\item High-Level Lifecycle (2004): Again agile-Based, its main aim being scaling agile development and focusing on putting forward the notion of AMDD approach.
\end{itemize}

After reviewing and analyzing each of them according to AMDD, MDD and Agility evaluation criteria; the author concludes that AMDD adopts different processes with no convergence between them, which can be a consequence of the lack of academic research in the area.

The final statement is that although AMDD is a promising field in a research and practical context, the author considers it not mature at all, stressing that it is still in its “infancy”. This paper can be used as an starting point in a AMDD research for creating, adapting or selecting AMDD processes.

\subsection{Agile Modeling Method Engineering}

This paper \cite{4} provides an overview on the practice of Agile Modeling Method Engineering or AMME. This practice mixes the use of modeling and agile methodologies, as a way to evolve modeling requirements all along the software life cycle and to connect more stakeholders and developers.

In other words, its goal is to outline the key characteristics of AMME as an emerging paradigm for tackling evolving modeling requirements emerging from a narrow domain and for specific
needs of modeling stakeholders.

The author stresses that this kind of methodology can be especially useful in an Enterprise Model paradigm in order to tackle complexity, changing requirements, modeling as knowledge representation and the generation of “conceptual model”-awareness at run-time.

In order to properly introduce this methodology, a framework and an architecture based on its principles are showed. The framework focuses on adaptability, extensibility, operability, integrability and usability along different framework modules such as tracking system, reusable asset repositories, prototyping environment and development channels. More details on the framework’s functionalities can be found in the next figure.

Finally this framework is instantiated within the Open Model Initiative Laboratory to support meta-modeling research projects and to test how would the AMME approach work under a real environment. 

As a conclusion, the author states the importance of accepting the AMME approach, because the Software Development field needs it as much as it did the Agile Manifiesto back in the day. Modeling
requirements should be the essential driver for modeling method engineering, and an approach based on AMME will enable agile response to evolving requirements, as well as traceability of change propagation across modeling method building blocks.

\subsection{Comparing and Contrasting Model-Driven Engineering at Three Large Companies}

Burden et.\,al. \cite{10} empirically investigate three companies, namely \emph{Ericson AB}, \emph{Volvo Cars} and \emph{Volvo Group}, which undergo a parallel shift to MDE and Agile Methods concerning their development methodologies.
They contribute to the broad empirical study by Hutchinson et. al. \cite{secondaryHutchinsonEmpirical} about the use of MDE in industry.
In general their observations validate the findings of Hutchinson et. al., e.\,g. MDE works best in narrow well-understood domains, it has to be driven bottom-up through developers and must be put on the critical path.
Though, they also refute some of the previous findings: 
Engineers are actually well-trained in MDE methodology by universities, code gurus appreciate code generators to reduce repetitive tasks, middle managers welcome MDE as it help minimizing risks and architects might find the complexity of MDE tools intimidating.
As a result Agile and MDE can yield synergies:
Agile profits from MDE by the enforced integration of stakeholders by abstracting away from the actual implementation and by reducing external dependencies because outsourced activities are brought back in-house through automation.
The success from MDE can profit from the bottom-up approach of Agile Methods.
A further interesting finding of \cite{10} is the importance of secondary tools. 
As MDE tools again can cause repetitive tasks, engineers naturally develop secondary tools, e.\,g. scripts to again automate this activities.

\subsection{Early Experience with Agile Methodology in a Model-driven Approach}

In \cite{11} Kulkarni et.\,al. report about their experiences on the introduction of Agile Methods in a MDE based software development process within \emph{Tata Consultancy Services}.
This company is specialized on the delivery of E-commerce solutions for a wide range of customers from different areas.
As their customers require individual solutions but with common requirements between customers, MDE turned out to be very effective as a majority of an application can be generated and only minor adjustments are required.
The development of their MDE platform used to be based on a traditional plan-driven approach. 
When they changed to an agile approach, they first suffered a loss of productivity and estimations were always wrong. 
After a few weeks, the process stabilized and they actually experienced an significant improvement of productivity and accuracy of estimates.
However, they experienced that a pure agile approach is not adequate for MDE, as sometimes in-depth analysis, exploration and research might be required in the development of the MDE tools.
As Agile promotes the delivery of executable artifacts and short timely intervals, these kinds of activities are barely feasible.
Therefore they adapted their approach by introducing Meta-Sprints, i.\,e. a longer sprint which is allowed to produced non-executable artifacts such as documentation, proof-of-concept prototypes etc.
Kulkarni et. al. really positive experiences with these adapted agile approach.

\subsection{Using Free Modeling as an Agile Method for Developing Domain Specific Modeling Languages}

Golra et.\,al. \cite{12} report on their experience about using Agile Methods for the development of Domain Specific Languages in a project for a governmental organization in France.
In spite of defining the language top down with strict semantics, they developed metamodel, model and graphical representations in parallel with loose coupling in between the artifacts.
They identified several positive finding on the effects of Agile within MDE.
Agile Methods within the development of graphical DSLs is a success story as it facilitates communication with domain experts and development requires flexibility both in tools and communication.
However, the authors point out that further conceptual studies in this area are required.
 
\section{Conclusion}

TODO ARTE write a conclusion %TODO ARTE

%----------------------------------------------------------------------------------------
%	REFERENCE LIST
%----------------------------------------------------------------------------------------

\bibliographystyle{acm}
\bibliography{lib}

%----------------------------------------------------------------------------------------

\end{document}
